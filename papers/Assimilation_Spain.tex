

%%%%%%%%%%%%%%%%%%%%%%%%%%%%%%%%%%%%%%%%%%%%%%%%%%%%%%%%%%%%%%
\section{Introduction}
Immigration has become an increasingly important phenomenon worldwide, with major implications for labor markets, social systems, and integration policies. According to the United Nations, in 2024 there were an estimated 304 million international migrants, representing about 3.7\,\% of the global population. This number has nearly doubled since 1990, when there were around 154 million migrants \citep{MigrationDataPortal}. Within Europe, recent inflows have also been substantial. In 2023, the European Union received 4.3 million immigrants from non--EU countries---of which Spain received almost 30 percent, making it by a large margin the destination with the largest inflow per capita among the group of large EU countries \citep{Eurostat2025}. These global and European developments highlight the growing relevance of understanding how immigrants integrate into host-country societies.  

A key aspect of immigrants' integration in the host country is their assimilation in the labor market, which refers to the process by which immigrants converge toward native-born workers in wages, employment, or occupational status. Understanding this process is essential for evaluating the economic contribution of immigration, designing integration policies, and assessing the long-run effects of demographic change.  

Classic studies emphasize the role of human capital accumulation, showing how immigrants acquire host-country-specific skills that gradually close their wage gaps with natives \citep{Chiswick1978}. Subsequent works highlight the importance of migrant self-selection and cohort quality \citep{Borjas1985, Borjas1995}, or the role of differential human capital investments, suggesting that lower initial earnings are associated with faster wage growth \citep{DuleepRegets2002}. More recent contributions have drawn attention to issue of selective out-migration, which can bias cross-sectional estimates upward if return migrants are negatively selected \citep{DustmannGorlach2016}. A further line of work highlights how labor market competition and imperfect substitutability between natives and immigrants shape assimilation patterns, implying that assimilation trajectories depend not only on individual skill acquisition but also on current labor market conditions \citep{AlbertGlitzLlull2025}.  

Spain provides a particularly instructive case for studying these dynamics. After centuries as a country of emigration, Spain became one of Europe’s main immigrant destinations during the late 1990s and 2000s. The foreign-born share of the population increased from less than 2 percent in 2000 to more than 14 percent in 2012, driven largely by inflows from Latin America, North Africa, and Eastern Europe \citep{OECD24, DeLaRicaGlitzOrtega2014}. These inflows coincided with a period of rapid economic growth, a deep crisis during 2008--2014, and a strong recovery thereafter. This unique boom--bust trajectory makes Spain an ideal laboratory for understanding how economic fluctuations shape assimilation. 

We employ longitudinal administrative data to study the assimilation patterns of the various immigrant cohorts in Spain that arrived during the 1990s and the 2000. While there is a notable downward shift in the level of relative wages across cohorts over time, once we condition on experience abroad and education level, the cross-cohort heterogeneity in initial wage gaps, which lie between 20 and 30 percent, become small. However, we find high variation in the speed of assimilation over time across cohorts, shaped by the economic cycle in Spain. Episodes of strong relative wage growth during boom periods alternative with falling relative wages during crises. These patterns are consistent with recent  evidence showing that immigrants’ wages in Spain are significantly more sensitive to the business cycle than those of natives \citep{Galvez22, escalonillaetal24}. In terms of assimilation in the long term, which we can thus far only assess for the arrival during the 1990s, our estimates imply that immigrants still suffer a wage gap to natives of around 10 to 15 percent after two decades. 

Examining assimilation patterns separately by origin region reveals that immigrants' average wage gaps are much larger for those from Africa, Latin America, Asia and EU countries that joined in 2004 or later, while immigrants born in EU15 countries have earnings that are on a par with those of natives and unaffected by the business cycle. 

This chapter concludes by discussing the role that factors not directly related to immigrants’ level of human capital might play for their relative earnings. Occupational barriers, such as the non-recognition of foreign credentials, may limit access to better-paying jobs \citep{bruckeretal21, Hermansenetal2025}. Differences in reservation wages can also matter: immigrants may be willing to accept lower pay due to weaker access to social insurance or because they derive higher real income from remittances or consumption in their origin countries \citep{Albert2021, DustmannKuSurovtseva2024}. However, most evidence suggests that these mechanisms lead to an earnings disadvantage for immigrants mainly during the first few years after arrival. Thus, the nature of the observed large wage gaps to natives of immigrants from outside the EU15 in Spain, still persisting even after decades in the country, remains to be fully understood and should be the subject of further research.
