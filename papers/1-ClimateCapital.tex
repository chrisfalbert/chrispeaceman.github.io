
Global warming is expected to reduce precipitation in subtropical regions, leading to large agricultural productivity losses (\citealt{ipcc2021}). A key question for understanding the economic costs of climate change is how labor and capital reallocate across space in response to these productivity shocks. Classic models predict that factors should flow away from affected regions toward areas with higher returns.\footnote{\cite*{corden1982booming}; \cite*{Matsuyama92}; \cite*{Krugman91}.} However, spatial frictions in labor and capital markets may constrain this adjustment process.\footnote{\cite*{GolbergPavcnik07}; \cite*{buera2011finance}; \cite*{gollin14agricultural}; \cite*{Munshi20}; \cite*{porzio22human}; \cite*{DonovanSchoellman23}.} 

In this paper, we develop a spatial equilibrium model that incorporates both bank branch networks and labor migration networks, and use it to guide an empirical analysis of how climate shocks propagate across space through these two channels. We test the predictions of the model using data on capital and labor flows across Brazilian regions. Brazil is suited for this analysis because its climate has already started experiencing the effects of global warming. We document an increase in meteorological dryness (combination of lower rainfall and higher temperatures) in the last decades relative to the previous century, as well as an increase in the frequency of droughts reported by municipalities to the federal government using newly digitized administrative data.\footnote{Climate models predict that global warming will increase precipitation in high and low latitudes but decrease it in middle ones, which encompass the majority of Brazilian regions (\citealt{ipcc2021}, page 645).} These changes have important implications for agricultural productivity, and have been largely heterogeneous across Brazilian municipalities, creating variation in local productivity shocks induced by climate that we can exploit to test the predictions of the model.

The objective of the model is to understand how regional productivity shocks induced by climate propagate across space. The model features two key channels of spatial integration: capital flows across regions, intermediated by banks operating a branch network in multiple locations, and labor migration across regions, governed by workers' location choices following pre-existing migrant networks. A distinctive feature of our framework is that firms face a cash-in-advance constraint---they must borrow from banks to finance a fraction of their wage bill before production takes place. This creates a link between local labor market conditions and credit markets. Banks allocate lending across locations based on expected returns, and when they face liquidity shocks in some locations, these may propagate throughout their branch network.

The model delivers an estimating equation that decomposes the response of local credit and employment to climate shocks into three components: (i) the direct effect of the local productivity shock; (ii) indirect effects via labor market integration, captured by the migration network; and (iii) indirect effects via capital market integration, captured by the bank branch network matrix. These theoretical predictions guide our empirical specifications and provide an economic interpretation for the reduced-form coefficients we estimate.

We capture regional climate shocks using the Standardized Precipitation and Evapotranspiration Index (SPEI), which measures deviations in drought conditions relative to a 100-year historical average (\citealt{vicente2010multiscalar}). We show that decadal changes in this measure are uncorrelated with initial municipality characteristics, which lends support to our differences-in-differences identification strategy. We construct measures of indirect exposure to climate shocks through capital markets using the structure of bank branch networks, following \cite{bustos2020capital}, and through labor markets using past migration flows, following \cite{borusyak2022understanding}.

We begin by documenting that persistent excess dryness causes large reductions in agricultural output. A municipality moving from the median to the 90th percentile of excess dryness experienced a decline in the value of agricultural production of more than 20\% over the last two decades. Effects are highly non-linear, with sharp reductions in output at the top of the dryness distribution but no significant effects of excess wetness. This confirms that climate shocks generate substantial local productivity losses.

We then study how these shocks propagate through bank branch networks. A key finding is that the short-run and long-run effects operate in opposite directions. In the short run, the banking sector provides insurance: regions experiencing droughts receive capital inflows and increased lending, funded by branches in other locations. This is consistent with consumption smoothing---farmers borrow against future income to cope with temporary shocks. However, this insurance provision exposes banks to climate risk. When shocks persist, affected borrowers cannot repay their loans, reducing bank liquidity. Due to frictions in the interbank market, banks then contract lending throughout their branch network.

The long-run results confirm this credit disruption channel. A municipality moving from the median to the 90th percentile of average excess dryness over the 2001-2010 period experienced a 16 percent decline in lending. Regions indirectly connected through bank networks also experienced significant credit contractions, with magnitudes about half the size of the direct effect.

Turning to labor markets, we find that climate shocks propagate through migration networks in the direction predicted by the model. Regions directly affected by excess dryness experience net outflows of migrants, while regions connected through pre-existing migrant networks experience net inflows. A municipality moving from the median to the 90th percentile of excess dryness experiences a 1.3 percentage points larger net outflow of migrants as a share of population. Conversely, a municipality at the 90th percentile of indirect exposure via migrants experiences a 0.76 percentage points larger net inflow rate.

These migration flows translate into employment effects. Directly affected regions experience a 2.5 percent decline in total employment, while regions connected via migration networks experience a 2.2 percent increase. Importantly, we find that the two channels of spatial integration generate opposite effects on destination employment: the migration network brings workers and expands employment, while the bank network transmits credit shocks and contracts employment. 


The combination of model and empirical analysis highlights the following key insights. First, as emphasized by previous literature (\cite{borusyak2022understanding}) ignoring indirect effects leads to attenuation bias when shocks are spatially correlated. Indeed, we show that the estimated direct effects on both capital and labor increase significantly when we control for spillovers. Second, the short-run insurance mechanism that banks provide---reallocating capital toward drought-affected regions---generates long-run vulnerabilities when shocks persist. Third, the model shows that indirect capital effects depend critically on how much banks rely on their own branch network versus the interbank market for funding. When shocks persist and borrowers default, banks with limited access to interbank funding contract lending throughout their branch network. We document that indirectly connected regions experience credit contractions about half the size of directly affected regions---consistent with the credit disruption channel embedded in our framework. Finally, we document that capital adjusts faster than labor: credit contractions in affected regions are an order of magnitude larger than employment effects. 


