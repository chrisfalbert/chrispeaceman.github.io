

Who buys imports? This question is fundamental to understanding the distributional effects of trade shocks, such as tariff increases, which incur greater costs for households spending more on imported goods. The limited existing evidence focuses almost exclusively on differential exposure to international trade by consumer income \citep{fajgelbaum2016measuring,cravino2017distributional,borusyak2021distributional,galle2023slicing,auer2024unequal}.

This paper is the first to explore how the consumer gains from trade vary with nativity. Immigrants' preferences and habits may differ from those of natives \citep{fernandez2009culture,atkin2016caloric,miho2023diffusion}, and to the extent that these differential preferences affect immigrants' demand for foreign goods, they will be differently impacted by trade shocks. Moreover, immigrants may also affect natives' gains from trade by increasing local import availability (\citealt{gould1994immigrant}, \citealt{head1998immigration}, \citealt{parsons2018migrant}), thereby shaping the distribution of the gains from trade across natives.

Quantifying the separate effects of immigrants on import demand and supply requires detailed data that simultaneously capture households' nativity, geographic location and the origin countries of their purchases. We construct, to our knowledge, the first dataset to contain all three elements by augmenting  U.S. grocery scanner data at the household level to include the origin country of both households and products.\footnote{We are unaware of any alternative data that provide information on both household nativity and import shares. For example, the Consumer Expenditure Survey, which may be combined with other datasets to measure import expenditure shares on cars as in \cite{borusyak2021distributional}, does not ask about respondents' nativity.}

We obtain three key results. First, immigrants have a stronger preference for imports than natives, leading them to accrue a disproportionate share of the gains from trade. Second, immigrants increase import supply by reducing the fixed costs of importation and increasing market size, thereby enhancing natives' gains from trade. Third, the majority of immigrant-induced local import expenditure is attributable to purchases made by immigrants themselves, rather than via spillovers to native import expenditure.

Regarding import demand, our data allow us to provide the first direct test of whether immigrants differ from natives in their import expenditure and by how much. Applying the general welfare formula of \cite{arkolakis2012new}, immigrants accrue consumer gains from trade within the grocery sector that are 42\% greater on average than those of native households. Furthermore, we show that within-county variation across households in import expenditure explains the vast majority of this differential, highlighting the role of preferences rather than sorting of immigrants into locations with high import supply. Our estimates suggest that, on average, immigrants spend 25\% more on all grocery imports and 127\% more on imports specifically from their own origin country when compared to an observationally identical native household living in the same U.S. county. Nativity is substantially more important than income in explaining variation in the gains from trade. The within-county difference in import expenditure between a household in the highest income quintile and an observationally identical household in the lowest income quintile is only $+$8.5\%, a third of the within-quintile gap between an immigrant and an observationally identical native household.

Regarding import supply, we develop and estimate a heterogeneous firms model of international trade \`a la \cite{melitz2003impact} extended to allow immigrants to affect local import costs. We leverage unique features of our dataset, including barcode-level price and variety counts by origin, in order to estimate the various channels through which immigrants affect local import supply. The leave-out push-pull instrumental variables developed by \cite{burchardi2019migrants} provide exogenous variation in the local population of immigrants by origin country. Our estimates imply that immigrants reduce the fixed costs of trade but have a negligible effect on variable trade costs.\footnote{That immigrants reduce fixed trade costs, but not variable trade costs, aligns with the previously untested assumption of \cite{peri2010trade}. We confirm this result within the NielsenIQ store-level scanner dataset: immigrants have no effect on import prices across stores, consistent with uniform pricing within large retail chains \citep{dellavigna2019uniform}. While we observe a positive effect of immigrants on import sales across stores, the immigrant effects documented in our household purchase records occur primarily through shopping trips to smaller retailers not present in the NielsenIQ retailer panel.}

We combine the estimated import demand and supply effects of immigrants within our full model and run three counterfactual exercises. In our first counterfactual, we consider what would happen to imports and native welfare in the absence of preference and trade cost effects of immigrants while holding the local population size fixed. National grocery import expenditure would fall by almost 9\%, akin to doubling prevailing tariffs applied to grocery goods. Immigrants' stronger preferences for imports explain 60\% of the immigrant-import elasticity, with the remaining 40\% attributable to changes in import supply common to all households. Approximately half of the import supply effect occurs via reduced fixed costs of trade, while the remainder is due to a novel channel we recover through our structural model: immigrants' heightened import preferences encourage foreign firm entry into local markets which in turn increases import variety for all households. The welfare gains for natives due to immigrant-induced trade---equivalent to about 0.1\% of grocery expenditure, or \$800 million---are 70\% less than one would infer from aggregate regional trade flow data and standard welfare formulas, such as \cite{arkolakis2012new}, as immigrants disproportionately increase local import expenditure via their own purchases.\footnote{Our estimated welfare effect is sizable. For comparison, \cite{brinatti2025firm} find that a 20\% increase in immigrants raises native welfare by 0.1\%, while \cite{caliendo2019trade} and \cite{hsieh2016global} estimate the welfare impact of the China shock to be between 0.03\% and 0.2\%.}

In a second counterfactual exercise, we show that the presence of immigrants yields significant consumption welfare gains for natives via an additional market size effect. If immigrants and their expenditure were to entirely disappear, native welfare from grocery consumption decreases by almost 8\%---equivalent to over \$59 billion in lost welfare nationally---primarily due to the loss of product variety resulting from smaller markets.\footnote{While this scenario is unrealistic, it provides a useful benchmark (comparable to the standard gains-from-trade exercise) by which to value immigration in terms of native expenditure.} These losses are highly concentrated among high-income and urban households.

Our final counterfactual simulates the heterogeneous impact of an increase in import tariffs leveraging our full model of import demand and supply. Specifically, we simulate a variable cost shock to all imported grocery goods, and recover the distributional welfare costs across households and counties. The average within-county difference in costs between immigrants natives is over six times larger than the across-county standard deviation in costs for native households. Variation in costs across geographies and income groups explain a fraction of the variation in costs between immigrants and natives. 

This paper provides the first direct evidence that immigrants, as consumers, derive substantially larger gains from trade than native households in the U.S. By encouraging foreign product entry into local grocery markets, immigrants also generate import consumption benefits for natives. Still, reductions in the immigrant population and higher costs of trade both incur relatively lower welfare losses for natives, especially those with lower education.

