


Prohibitively high housing costs in the most prosperous regions in the United States, such as the Bay Area or New York City,  prevent many workers from moving to these locations and accessing their labor markets. In a recent paper, \cite{Morettietal16} argue that the constrained housing supply in many of these large and highly productive metropolitan areas limits ``the number of workers who have access to such high productivity,'' something that the authors refer to as the spatial misallocation of labor.

Despite their high cost and limited supply of housing, it is well known that prosperous cities attract many foreign-born workers. Indeed, a closer look at the data reveals that there are large differences between natives and international migrants in terms of how likely they are to choose to locate in expensive, high-productivity cities. While 4.8\% and 1.6\%  of natives lived in New York City and the Bay Area in 2019, respectively, these figures are 12.7\% and 5.1\% for international migrants. In this paper, we argue that, due to their different consumption patterns, immigrants have stronger incentives to locate in expensive and highly productive cities than natives. We also quantify how much immigration, through this consumption channel, contributes to reducing the spatial misallocation of labor. 

We base our argument on the following idea. Immigrants tend to spend large fractions of their income in their home countries. Many send remittances to family members left behind, plan on returning, or simply spend their leisure time at home \citep{DustmannMestres10, DustmannGorlach16}. This means that they take into account not only the prices in the location where they live but also the prices in their home countries. We argue that, relative to natives, immigrants are less deterred by the high housing costs of the most productive cities because they consume less locally and consume more abroad. Hence, immigrants concentrate more in high-productivity cities, which helps to alleviate the spatial misallocation of labor identified in prior literature.  


In the first part of the paper, we document three strong empirical regularities that support our argument. First, we show that immigrants concentrate disproportionately in metropolitan areas with high living costs, where, as it is well known in the urban economics literature, nominal wages and productivity tend to be higher \citep{Combesetal14,Glaeser08}. This fact also holds when we compare immigrants' and natives' location patterns within finely defined education, experience, and occupation groups. It is also robust to instrumenting local housing prices by exogenous determinants of housing costs, such as available land or estimates of the local housing supply elasticity, and when controlling for city size. Quantitatively, our results suggest that the \textit{relative} probability of finding an immigrant is around eight times higher when local prices double. 
 
Second, we show that there is a marked heterogeneity across immigrant groups. When home-country prices are lower, immigrants may prefer to consume a higher fraction of their income at origin. If so, this raises the incentives for immigrants from countries with low prices to locate in destinations with high prices. We use cross-origin and, arguably exogenous, within-origin variation in real exchange rates to document that when exchange rates are lower, immigrants' concentration in expensive cities is stronger. We also show, using Matricula Consular data on state-to-state migration flows from Mexico to the United States, that Mexican immigrants from poorer states, where presumably price levels are lower, tend to disproportionately migrate to the richest and most expensive US states. All these results also hold when we flexibly control for both city and immigrant network size. 

Third, we provide evidence that immigrant households consume around 15\% less locally than comparable native households. Indeed, both when we use Consumer Expenditure Survey data and focus on local consumption or when we investigate housing consumption using Census and American Community Survey (ACS) data, we find that, conditional on being in the same city, immigrants spend less than natives with similar income levels, family sizes, and education. Immigrants tend to live in cheaper and smaller apartments, or, in brief, demand lower overall housing services than natives. We also document that these patterns are stronger for immigrants from countries with lower prices.   


In the second part of the paper, we show how these empirical regularities can be readily explained by a standard spatial equilibrium model in which immigrants consume -- for example, via remittances -- a part of their income in the origin country. Intuitively, while in standard spatial equilibrium models a (marginal) native is indifferent between one location and an alternative one that is twice as expensive, as long as wages are also twice as high, immigrants' smaller share of local consumption implies that they prefer the high-wage, high-price city. As a consequence, immigrants concentrate in expensive cities, where they consume less housing and other non-tradable services, as we see in the data. Some degree of substitutability between home-country and destination goods makes this mechanism stronger for immigrants coming from cheaper countries, which is in line with the data on immigrant concentration both when we compare location patterns across countries of origin and when we relate them to fluctuations in exchange rates. 

We argue that our mechanism, rather than alternative hypotheses, generates these patterns. The fact that the wages of immigrants relative to natives are lower in expensive cities -- controlling for observable characteristics -- suggests  that these patterns are not driven by differential relative demands across cities. Complementarities among different types of workers are also unlikely to explain our results, since they also hold when we condition on particular education levels, and when we we exclude occupations that prior literature has identified as complementing particularly well the high-skilled jobs that have been thriving in large cities \citep{AutorDorn13}. Our results are also unlikely to be driven by immigrant networks. We perform several robustness tests, which suggest that while immigrant networks have power to explain where immigrants locate, our mechanism retains strong empirical bite. We also incorporate the role of immigrant networks in our theoretical model, acting as an amenity shifter specific to the country of origin. More generally, as we argue in more detail below, potential alternative mechanisms have a hard time explaining the systematic relationship between immigrant concentration heterogeneity and home-country prices, and, at the same time, the consumption patterns that we document. 


In the third part of the paper, we estimate the model by matching the distribution of immigrants relative to natives in 1990, using variation across metropolitan areas and across countries of origin conditional on the relative size of local immigrant networks. The estimation identifies two key parameters. First, our estimates imply that immigrants' expenditure share in the home country would be around 13\% if destination and origin prices were equal. Second, we estimate an elasticity of substitution between consuming locally and in the origin of around 3. Thus, due to the substitution effect, immigrants increase their expenditure share in the home country when it becomes cheaper relative to the destination country. We validate our estimation by showing that the model fits the data well and does a good job at predicting two additional non-targeted moments: the variation in the consumption of housing across origins and the immigrant inflows and native population growth patterns across cities between 1990 and 2000 observed in the data.   

Finally, we use our model to study the extent to which immigration alleviates the spatial misallocation of labor. For this, we compute the spatial distribution of economic activity that would prevail if immigrants' consumption behavior was identical to that of natives and compare it to the baseline equilibrium. This exercise suggests that aggregate output and welfare in 2000 were around 0.47\% and 1.58\% higher, respectively, thanks to immigration. These effects are slightly larger than artificially lowering the housing supply elasticities in New York City, San Francisco, and San Jose -- the three cities that \cite{Morettietal16} identify as the main culprits for the spatial misallocation of labor -- to that of the median city. Moreover, we quantify the impact of immigrants' different preferences on the distribution of workers across cities. We find that some of the most productive cities are 20-30\% larger than they would be without immigration -- again, similar in magnitude to the effect of lowering the housing supply elasticities in New York City, San Francisco, and San Jose to that of the median city.  

Overall, this paper contributes to the literature in two ways. First, it provides a new theory for immigrants' location choices within host economies that is well supported by a large set of empirical facts, many of which have not been documented in previous research. Other existing theories of immigrants' location choices can at best explain a subset of these facts, but none is able to explain all of them in a unified and parsimonious framework. Second, we use our framework to better understand how immigration is shaping economic activity across locations and in general equilibrium, which typically escapes empirical accounts of immigration in the existing literature that mostly rely on difference-in-difference types of comparisons. 

