We analyze a multiyear, multicountry entrepreneurship survey with more
than one million observations to identify startups with low and high growth
potential. We confirm the validity of these ex ante measures with ex post firm-
level information on employment growth. We find that negative aggregate
financial shocks reduce all startup types, but their effect is significantly
stronger for startups with high growth potential, especially when GDP
growth is low. Our results uncover a new composition of entry channel that
significantly reduces employment growth and is potentially important for
explaining slow recoveries after financial crises.


A well-established literature documents the importance of financial
frictions for entrepreneurial entry and for the survival and growth of
newfirms(seeHoltz-Eakin,Joulfaian,andRosen1994;Blanchflowerand 
Oswald 1998; Corradin and Popov 2015; Schmalz, Sraer, and Thesmar
2017; Adelino, Schoar, and Severino 2015, among others). However, less
is known about the relation between financial factors, the decision of
whattypeofbusinesstostart,andtheexpostperformanceofnewfirms.
Haltiwanger et al. (2016) show that while most new firms grow slowly, a
small fractiongrows very rapidly, driving ahigher mean netemployment
growth for younger firms than for older firms. Pugsley, Sedlacek and
Sterk (2018) argue that such heterogeneity is primarily driven by the ex
ante characteristics of these startups rather than by the ex post shocks
they face during their lifetime.
Are these ex ante decisions of the entrepreneurs important for the ex
post ability of their businesses to create jobs? And do financial factors
affect these ex ante decisions? This paper provides new evidence and an
answer to these questions by combining multiple data sources. Our main
data set is drawn from the Global Entrepreneurship Monitor (GEM),
a multicountry survey of entrepreneurial decisions that allows us to
identify heterogeneous startup types. We use a sample of this survey
that ranges from 2002 to 2013 and includes a total of approximately
one million individual-level observations from 21 OECD countries. We
merge this data set with firm-level data, which allows us to measure the
ex post performance of these different startup types, and we employ a
country-specific business cycle indicator (gross domestic product [GDP]
growth) and several macroeconomic indicators of financial conditions,
which have been shown to strongly affect the availability of credit to
households and businesses.
Three features make the GEM data set particularly suited for our
purpose. First, it includes an individual’s personal characteristics,
such as age, gender, education, income bracket, and entrepreneurial
experience. Thus, we can study the dynamics of startups while
controlling for the quality of the pool of potential entrepreneurs. Second,
it is designed to be representative of a country’s population and contains
harmonized data across countries. Poschke (2018) shows that the firm
size distribution obtained from survey responses of entrepreneurs in the
GEM matches remarkably well with that obtained from administrative
data sources. Third, it includes survey questions to ascertain the
expected employment growth of new startups and the innovative nature
of the products and services that will be offered; we use the survey
questions to identify startups with high growth potential.
To formalize the intuition behind the relation between financial
frictions and startup selection, we develop a stylized partial equilibrium
model, in which new entrepreneurs start a business by paying an initial
sunk cost that is financed partly with their own wealth and partly with
debt, for which they pay a premium over the market interest rate. This
premium reflects the excess cost of external finance caused by financial
frictions. The entrepreneurs can choose between two different types
of businesses: type 1 represents a business model that is reliable and
immediately profitable but with limited growth potential, for example,
a business model in which the entrepreneur decides to provide well-
established services and/or products in well-known markets. Type 2
represents the decision to provide a newer product or service and/or
one in less well-known markets. The type 2 business is initially not
as productive as the type 1 business but has a much larger growth
potentialinthemedium-longterm.Theentrepreneursareheterogeneous
in their ability to manage these different businesses: in equilibrium, for
themarginalentrepreneurwhoisindifferentbetweenthetwotypes,type
2 has lower profitability in the short term and higher profitability in the
long term. It follows that at the margin, it takes longer to repay the
initial debt to finance a type 2 startup, and its value is more sensitive to
short-term increases in the cost of external finance than that of a type
1 startup.
These results imply that, conditional on aggregate conditions and the
quality of the entrepreneurial pool, an increase in the excess cost of
finance will reduce the number of all startups and the number of type
2 by relatively more than that of type 1 startups. Moreover, the results
imply a financial accelerator channel that operates via the creation
of new startups. By reducing the disposable income of entrepreneurs,
a decline in GDP growth increases the need for external finance and
amplifies the negative effects of financial shocks relatively more for type
2 than for type 1 startups.
To test these predictions, we identify type 2 startups in the GEM
data set as those businesses for which the entrepreneur is expecting high
future employment (relative to the average employment of established
firms in the same country and sector). A key part of our analysis is
that we verify whether this ex ante entrepreneurial selection of types
is able to predict faster ex post firm growth. Conducting this type of
test using only the GEM survey, which is a repeated cross-section, is
unfeasible. Therefore, we match it at the two-digit sector level with
a sample obtained from the Sistema de Análisis de Balances Ibéricos
(SABI) comprising all new firms founded since 2003 in Spain. Despite
being limited to Spain, the matched firm-level data set is sufficiently
representative for our purposes. Spain is the country with the most
extensive coverage in GEM, with more than 200,000 observations.
Indeed, all the main results we later obtain from the entire data set
are also confirmed when considering only the Spanish GEM surveys.
The matched sample includes 46 two-digit sectors and 226,954 firm-
year observations. We link each firm in SABI with the share of startups
with high growth potential in its sector in the year it was founded. We
interpret this value as the probability that this firm is a high-growth 
firm. We find that the higher this ex ante probability is, the smaller the
initial employment for new firms but the faster the employment growth
over time: this faster employment growth results in the high-growth
firms having a significantly larger size from 6 years of age onward. This
result is robust to controlling for sector-year fixed effects and for the
aggregate conditions at the time of the firms’ entry, and therefore, it is
not driven by sector- or time-specific factors. In other words, this finding
provides a positive answer to our first question. The ex ante decisions of
the entrepreneurs on the type of startup significantly affect the ex post
ability of these businesses to create jobs.
After verifying the validity of our empirical measure of ex ante high
growth potential, we provide an answer to our second question by
testing the predictions of the model. Financial shocks are measured
by fluctuations in the excess cost of external finance. Our preferred
indicator is the Gilchrist and Zakrajsek (2012) bond spread for financial
institutions. Using additional data on European countries from Gilchrist
and Mojon (2016), we compute the indicator for the United States,
Spain, Italy, France, and Germany. Gilchrist and Mojon (2016) show
that such spreads are good proxies for credit availability to households
andfirmsandhavestrongpredictivepowerfortherealeffectsoffinancial
crises. Therefore, they are ideal measures of the intensity of financial
frictionsaffectingnewstartups.Wealsocheckthattheresultsarerobust
tousinganalternativemeasureoffinancialfrictions,suchasthefinancial
distress indicators of Laeven and Valencia (2013) and of Romer and
Romer (2017).
Our main results confirm the model’s hypotheses. We find that
conditional on GDP growth and individual characteristics, all startups
are negatively affected by financial shocks but high-growth startups are
affected much more than low-growth ones. Moreover, we find a strong
interactionbetweenfinancialfrictionsandGDPgrowth:withlowerGDP
growth, the negative effect of financial shocks on startups with high
growth potential becomes more amplified than the negative effect of
financial shocks on low-growth startups.
As an additional test of our hypothesis, we consider two indicators
often used in the literature to determine the sectors that are most likely
to face financial frictions: the external financial dependence indicator
(Rajan and Zingales 1998) and an indicator of intangibility (the share
of intangible over total assets; see Falato, Kadyrzhanova, and Sim
2013; Caggese and Perez 2017). The model predicts that startups
in sectors with higher indicators should be more negatively affected
by financial shocks, and we confirm these predictions in the data.
Furthermore, our results are also confirmed when we control for the
risk-free interest rate or variations in the term premium, which could
differently affect the expected value of the different startup types, and
when we control for additional individual-level characteristics, such as
expectations about future business opportunities, income category, and
previous entrepreneurial experience.
Taken together, our results strongly support the view that financial
frictions have different effects on the entry of firms with high growth
potential and that this composition of entry channel is important for
explaining slow recoveries after financial crises, which imply highly
persistent output losses, as shown by Cerra and Saxena (2008).
Abstracting from the general equilibrium effects on wages and prices,
our results imply that a recessionary period accompanied by a one-
percentage-point increase in the bond spread changes the nature of
newly created firms such that after 10 years, the employment level in
these firms is on average 4.3 percent lower.