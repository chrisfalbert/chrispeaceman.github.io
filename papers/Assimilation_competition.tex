
The evolution of immigrants' wages over time relative to those of natives is widely used as a measure of successful integration. Documenting this process, and understanding its impediments and facilitators, is essential for migration policy design and has been the subject of an extensive literature in economics. Following the seminal work by \citet{Chiswick78} and \citet{Borjas85}, most studies focus on two specific aspects of immigrants' wage assimilation: the initial immigrant-native wage gap and the way in which this wage gap changes as immigrants spend time in the host country. The first one is generally viewed as a proxy for the ``quality'' of immigrants in terms of their human capital upon arrival, the second one as reflecting immigrants' accumulation of host-country-specific skills, or the ``speed of assimilation''. For the United States, it has been documented that the initial wage gap between newly arriving immigrants and natives has widened significantly since the 1960s and that the speed of wage assimilation has simultaneously declined \citep{Borjas15a}, leading to the view that immigrants have become more negatively~selected~over~time.

In this paper, we argue that changes in the quality of more recent cohorts are only part of the story. In particular, we show that the increasing size of immigrant arrival cohorts, and the resulting changes in labor market competition, have an important effect on relative wages in equilibrium. The intuition behind this new mechanism is straightforward. When immigrants and natives are imperfect substitutes in the labor market, for example because of a comparative advantage in different occupations, their relative wages are partly determined by the aggregate supply of foreign workers in the economy. Increasing immigrant inflows, such as those observed in the United States over the last half century, then raise labor market competition more for immigrants than for natives, driving their wages apart and thus directly affecting wage assimilation. This effect is further amplified if technological progress leads to an increase in the relative demand for those skills that are relatively more abundant among natives. 

%%%%%%%%% ADJUST NUMBERS HERE %%%%%%%%%%%%%

Our findings suggest that, in the United States, changes in labor market competition alone explain about one fifth of the observed increase in the average immigrant-native wage gap across arrival cohorts since the 1960s. This figure rises to about one third once shifts in relative skill demands are accounted for as well. These aggregate findings conceal substantial heterogeneity across different immigrant groups. 

%%%%%%%%%%%%%%%%%%%%%%%%%%%%%%%%%%%%%%%%%%%

The theoretical basis of our empirical analysis is a production framework in which natives and immigrants supply two types of skills: general skills that are portable across countries, and specific skills that are particular to the host country. Upon arrival, immigrants are endowed with the same amount of general skills as observationally equivalent natives but only a fraction of their specific skills. Over time, immigrants accumulate further specific skills at a (usually) faster rate than natives, inducing wage convergence. The aggregate amounts of general and specific skills supplied in the economy are combined by a constant elasticity of substitution (CES) production technology. Technological change is allowed to increase the relative demand for either of the two types of skills. Equilibrium skill prices are competitively determined, which implies that relative skill prices depend on aggregate skill supplies, and workers are paid according to the skill bundle they supply. In our framework, imperfect substitutability between immigrants and natives arises as a consequence of their different skill sets. Since immigrants disproportionately supply general skills, increasing immigrant inflows shift relative prices in favor of specific skills, widening the wage gap between immigrants and natives. This effect is particularly pronounced in the early years after arrival when immigrants still have relatively few specific skills. In later years, in contrast, immigrants' skills are already more similar to those of natives, making their relative wages less responsive to changes in equilibrium skill prices. Whether immigration-induced changes in labor market competition increase or decrease the speed of wage assimilation depends on the precise magnitude and timing of the immigrant inflows as well as the immigrants' skill accumulation profiles.

We fit our model by nonlinear least squares (NLS) using data from the U.S. Census and the American Community Survey (ACS) that cover the period 1970 to 2019. We exploit individual-level variation to estimate the parameters determining the skill accumulation process and identify the technology parameters of our production function from relative wage differences across labor markets (defined by states and time). Based on the results from this estimation, we then decompose the observed changes in the initial wage gap and relative wage growth between the 1960s and 1990s cohorts into three components: the labor market competition effect, a demand effect driven by skill-biased technological change, and a residual component that reflects changes in cohort quality, both due to changes in education and country of origin, and due to changes in unobservable skills. 

%%%%%%%%% ADJUST NUMBERS HERE %%%%%%%%%%%%%

Our results show that immigration-induced increases in labor market competition can explain 16.4 and 50.3 percent of the increase in the initial relative wage gap of the 1970s and 1980s cohorts relative to the 1960s cohort. Shifts in relative skill demand account for an additional 17.2 and 45.4 percent. With more years spent in the United States, these effects diminish since immigrants become closer substitutes to natives. Averaged over time, the competition effect alone accounts for 13.5 and 21.3 percent of the increase in wage gaps relative to the 1960s cohort. For the 1990s cohort, it accounts for 19.2 percent. When additionally considering the effects due to changes in relative skill demand, these figures increase to 24.3, 38.8 and 44.8 percent, respectively. The remaining differences can be attributed to decreasing cohort quality and fully explained by changes in immigrants' education and origin composition. Conditional on these two observable characteristics, our findings suggest that immigrants have become more positively selected in terms of unobservable skills. Through a series of robustness checks, we show that our findings are largely unaffected by selective return migration, undercounting of undocumented immigrants, network effects, alternative functional form assumptions, endogenous immigrant location choices, and alternative formulations~of~the~production~function. 

We also find that these effects are highly heterogeneous across immigrant groups. For Mexican high school dropouts, for example, rising labor market competition increased the initial wage gap by more than 10 log points between the 1960s and 1990s arrival cohorts. For Western college graduates, in contrast, it had essentially no impact. Our results further show that rising labor market competition cannot explain the observed slowdown in average wage assimilation across cohorts, even though for some groups, such as Latin American high school graduates, it did inhibit long-run wage assimilation. Overall, we find that 33.5 percent of the total variation in the initial wage gaps across different types of immigrants and 29.5 percent of the variation in long-term assimilation rates can be attributed to changing labor market competition.

%%%%%%%%%%%%%%%%%%%%%%%%%%%%%%%%%%%%%%%%%%%

Our paper contributes first and foremost to the large literature that studies the wage assimilation of immigrants. After the pioneering work by \citet{Chiswick78} and its crucial extension to repeated cross-sectional data by \citet{Borjas85, Borjas95a}, numerous studies have analyzed the wage assimilation of immigrants in different host-country settings and time periods (see \citealp{DustmannGlitz11}, and \citealp{DustmannGorlach15}, for surveys of the international literature). For the United States, an extensive body of research has documented the widening wage gaps across arrival cohorts as well as the declining speed of wage convergence between immigrants and natives (see \citealp{Borjas14}, and \citealp*{Cadenaetal15}, for surveys of the U.S. assimilation literature). Contrary to most of this literature, our paper shows that these empirical regularities are not driven by changing immigrant cohort quality alone but that an important part can be explained by increasing cohort sizes and labor market competition, as well as changes in the relative demand for host-country-specific skills. 

Several papers in the literature have critically assessed some of the key assumptions underlying the estimation and interpretation of immigrants' wage assimilation profiles. \citet*{Bratsbergetal06} show that changing aggregate labor market conditions (measured by local unemployment rates) affect immigrants and natives differentially, leading to an upward bias in the assimilation rates obtained from the standard specification in the literature. To the extent that such changes in aggregate conditions are reflected in relative skill prices, our framework incorporates their differential effect on immigrant and native workers. \citet{DuleepRegets13} document a strong inverse relationship between immigrants' earnings at entry and subsequent wage growth, which they explain by higher investment in human capital of immigrants that arrive with less transferable skills. \citet*{DustmannKuSurovtseva24} show that variations in the real exchange rates between home and host countries are an important driver of immigrants' initial wage gaps and subsequent career trajectories. \citet{Lubotsky07} and, more recently, \citet{AkeeJones24} and \citet{RhoSanders21}, use  longitudinal administrative data matched with U.S. survey information to show that selective return migration may significantly bias estimated relative wage profiles, a conclusion also supported by the findings in \citet{Hu00} and \citet*{Abramitzkyetal14}.\footnote{\,For a systematic treatment of the issue of selective return migration in the context of immigrants' wage assimilation, see \citet{DustmannGorlach15}.} Due to the long period covered by our analysis, we cannot account for selective return migration as comprehensively as these studies do, but we show by means of three separate robustness checks that this issue is unlikely to affect our main conclusions. 

Some papers in the literature highlight the importance of skill prices and occupational mobility for the wage assimilation of immigrants. \citet{LaLondeTopel92} find that the relative earnings of immigrants are sensitive to persistent changes in wage inequality in the United States. In particular, since immigrants tend to be less skilled than natives, the rising returns to skills in the 1970s increased wages of the average native by more than those of the average immigrant. \citet{Lubotsky11} performs a similar analysis including more recent arrival cohorts using longitudinal social security data. \citet{LessemSanders20} highlight the important role of occupational upgrading for immigrant wage growth and quantify the potential benefits from removing barriers to occupational mobility. Neither of these studies, however, considers labor market competition due to imperfect substitutability between immigrants and natives as a key driver of relative~wage~profiles.

Our work is also related to a small number of papers that emphasize the link between immigrants' labor market outcomes and the size of different arrival cohorts. \citet{Beaman12} analyzes the importance of social networks for immigrant wage dynamics, exploiting exogenous variation from a refugee resettlement policy in the United States. She finds that an increase in the number of contemporaneously resettled social network members worsens the labor market outcomes of immigrants, whereas an increase in the number of tenured network members improves them. These results are consistent with our finding that immigrants who arrive around the same time are relatively substitutable due to their similar skill sets, but that the substitutability between different cohorts declines the further apart their respective times of arrival. In line with this observation, \citet*{DAmurietal10} find evidence for imperfect substitutability between ``new'' (0--5 years since arrival) and ``old'' (more than 5 years since arrival) immigrants in Germany, suggesting that new immigrant inflows have larger wage impacts on more recent immigrants than on older immigrants, consistent with earlier results for the United States reported in \citet{LaLondeTopel91}. While our analysis does not focus on the wage impacts of immigration per se, our theoretical framework fully captures, and indeed generalizes, these patterns of imperfect substitutability across different arrival cohorts. It also builds on the idea that the wages of natives and immigrants with different tenure in the country are differentially affected by new immigration.

In contemporaneous work, \citet{GaleoneGorlach22} study immigrant wage progression through the lens of an asymmetric nested CES production function in which each nest represents either immigrant workers with a specific number of years of residence in the U.S. or natives. As immigrants move across nests, their skill efficiency and substitutability with other factor inputs change, which jointly determines their wage growth. Using Census and ACS data for the years 2000 to 2018, the authors show that, while immigrants' skill efficiency increases significantly over time, part of the associated wage gains are offset by immigrants becoming increasingly substitutable with natives and earlier immigrants. Similar to our paper, their analysis highlights that observed wage profiles of immigrants generally reflect both genuine skill accumulation and changes~in~aggregate~factor~supplies.

Finally, our analysis is closely linked to the large literature on the labor market impact of immigration (see e.g. \citealp{Cadenaetal15}, and \citealp*{Dustmannetal16}, for surveys of this literature). One important insight that has emerged over time in this research area is that immigrants and natives are often not perfect substitutes in the labor market, even conditional on observable skills such as education and experience (see e.g. \citealp{PeriSparber09}, \citealp{OttavianoPeri12}, \citealp*{Manacordaetal12}, and \citealp{Llull18a}). As a result, new immigrant inflows have a less detrimental impact on natives than on previous immigrants, with much of the literature seeking to estimate the magnitudes of these relative wage effects.\footnote{\,In the context of internal migration in the United States, \citet{Boustan09} shows that, due to imperfect substitutability between black and white workers, the large black migration flows from the South to the North in the mid-20th century widened the racial wage gap in the North by 5 to 7 log points.} On closer inspection, however, the finding of imperfect substitutability generates a conceptual tension between the wage assimilation literature and the labor market impact literature. Even though both literatures study essentially the same outcome variable -- the relative wages of immigrants and natives -- they each account for its main determinants in very distinct and partial ways. While the traditional assimilation literature abstracts from aggregate factor supplies as possible drivers of relative wages, the impact literature usually does not, or only very rudimentarily, allow for immigrants' skill accumulation and evolving substitutability with other factor inputs. Our theoretical framework synthesizes these two long-standing and influential literatures, showing in an intuitive way how aggregate factor supplies and individual skill accumulation interact to give rise to heterogeneous wage profiles across workers.
