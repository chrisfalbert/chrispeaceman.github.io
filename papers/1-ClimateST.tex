Climate change is expected to generate persistent declines in agricultural productivity in many developing countries, particularly those located in tropical and subtropical regions (\citealt{ipcc2021}). Understanding how economies adjust to these shocks is central to assessing the long-run costs of climate change. Classic trade and spatial equilibrium models predict that declining agricultural productivity should trigger a reallocation of labor and capital away from agriculture and toward sectors and regions with higher returns.\footnote{\cite*{corden1982booming}; \cite*{Matsuyama92}; \cite*{Krugman91}.} In practice, however, this adjustment process may be constrained by frictions in labor and capital markets, limiting economies' ability to adapt efficiently.

In \cite{MestieriBustosPonticelliAlbert2025}, we study how climate-induced productivity shocks propagate across space through labor migration and bank branch networks. That paper develops a quantitative spatial equilibrium framework and shows that persistent increases in excess dryness in Brazil generate sharp contractions in credit not only in directly affected regions but also in financially integrated regions. While the banking system provides insurance against short-run weather shocks, persistent droughts lead to a disruption of credit supply that propagates through bank networks, constraining capital reallocation across space. In parallel, climate shocks induce large migration flows along pre-existing migrant networks, reshaping local labor markets.

This paper builds on \cite{MestieriBustosPonticelliAlbert2025} by studying how these spatial labor and capital flows translate into sectoral reallocation \emph{within} and \emph{across} regions. While our previous work focuses on aggregate employment, migration, and credit, the present paper asks a more granular question: how do climate shocks reshape the allocation of labor and capital across agriculture, manufacturing, and services, both locally and in regions connected through factor markets? Answering this question is essential for understanding whether climate change accelerates or hinders structural transformation, and whether factor market frictions distort the sectoral allocation of resources.

We address this question by combining a simple small-open-economy framework extending the classic Ricardo-Viner model\footnote{For a textbook discussion, refer to \cite{dixit1980theory}.} with detailed data on sectoral employment, migration, and credit flows across Brazilian municipalities. We show that persistent increases in excess dryness generate a large contraction in lending to both agricultural and non-agricultural firms, not only in directly affected municipalities but also in financially integrated regions. Consistent with this contraction in credit, manufacturing employment falls sharply in regions exposed to climate shocks through bank branch networks. Turning to labor markets, we find that regions directly affected by excess dryness experience large employment declines in the agricultural and service sectors, while local manufacturing absorbs only a fraction of displaced workers and most adjustment occurs through out-migration, with climate migrants disproportionately allocated to agriculture and services rather than manufacturing in destination regions. Together, these findings indicate that while climate change triggers substantial reallocation of labor and capital, spatial labor market frictions and financial constraints severely limit the ability of manufacturing to absorb displaced factors, distorting the adjustment process predicted by frictionless models.


In the neoclassical framework guiding our empirical analysis, factor allocation across the two tradable sectors -- agriculture and manufacturing -- depends on comparative advantage, which is driven both by relative productivity and factor abundance. In turn, the employment share of the non-tradable service sector depends on local demand, which is a function of local income per capita. A local increase in dryness reduces agricultural productivity, which worsens comparative advantage of local agriculture relative to local manufacturing. In addition, it reduces land rents and the local demand for services.  Thus, labor and capital reallocate away from both agriculture and services into local manufacturing.  

The model also generates predictions for the \emph{indirect effects} of excess dryness in regions integrated with areas suffering droughts through goods, labor or capital markets. First, because all regions are price takers in international markets, there are no spillover effects through goods markets. Still, the model generates predictions for the effects through labor and capital markets in regions which are the destination of factor inflows from other regions.  We think of these inflows as permanent changes in the supply of labor or capital that are exogenous from the point of view of the destination region. An inflow of mobile production factors increases the scarcity of the fixed factor land and thus reduces the comparative advantage of the agricultural sector. In addition, a higher scarcity of land implies lower relative income in traded goods and a lower relative demand for services.  As a result, the indirect effect of dryness through labor or capital inflows is an increase in the manufacturing employment share of both factors. 

We confront these predictions with rich administrative and census data from Brazil, which allow us to track both sectoral outcomes and spatial factor flows. As in \cite{MestieriBustosPonticelliAlbert2025}, we combine municipality-level Population Census data with detailed information on bank branch balance sheets from the Central Bank of Brazil (ESTBAN). The Census data allow us to measure employment, sectoral composition, and migration flows across municipalities over time, covering both formal and informal workers. The bank branch data provide a comprehensive picture of credit allocation across regions and sectors, and allow us to construct measures of financial integration based on the structure of banks' branch networks. 

A novel contribution of this paper is the use of matched employer--employee data from the Brazilian Annual Social Information System (RAIS) to study the microeconomic mechanisms underlying spatial labor reallocation. RAIS covers the universe of formal employment relationships in Brazil and allows us to track workers across firms, sectors, and municipalities over time. Relative to Census data, RAIS offers three key advantages. First, it provides firm-level identifiers, which allow us to observe directly how migrant workers are allocated across firms within the same destination labor market, rather than only across sectors or municipalities. Second, it contains detailed employment histories that make it possible to construct firm-specific measures of exposure to migrant networks based on past worker flows, which we interpret as revealing spatial labor market frictions in hiring and matching. Third, the panel structure of RAIS allows us to control flexibly for firm fixed effects, isolating within-firm responses to climate-driven labor supply shocks. These features enable us to directly test whether manufacturing firms face systematically higher barriers to absorbing climate migrants than firms in other sectors.

Our empirical analysis yields four main findings. First, we document that persistent increases in excess dryness generate a large contraction in lending to both agricultural and non agricultural firms. This contraction operates both directly, in municipalities experiencing higher excess dryness, and indirectly, in municipalities financially integrated with affected areas through bank branch networks. In contrast, municipalities exposed to excess dryness through migration links experience an increase in lending, which is stronger outside the agricultural sector. This pattern is consistent with the model prediction that an inflow of labor into a region leads to a relatively stronger expansion of the local manufacturing sector.

Second, turning to the sectoral allocation of labor, we find that, consistent with the neoclassical benchmark, regions directly affected by persistent increases in excess dryness experience a sharp contraction in agricultural employment and a decline in service sector employment. While this leads to spatial reallocation of workers through out-migration and an overall employment decline, as extensively studied in \cite{MestieriBustosPonticelliAlbert2025}, we find that local manufacturing expands by absorbing a part of the displaced workforce, indicating that within-region sectoral reallocation broadly follows the efficient adjustment path predicted by the model. 

Third, regions financially connected to those hit by dryness experience a sharp decline in manufacturing employment. This result is consistent with the fall in lending we observe in such regions and the model prediction that a fall in overall capital implies a reallocation of labor out of manufacturing and into the remaining sectors.

Fourth, spatial reallocation of labor across regions is severely distorted by factor market frictions. Migrants from drought-affected regions are less likely to reallocate into the manufacturing than other sectors in destination municipalities, which is inconsistent with the model prediction that manufacturing should expand when labor inflows increase. Instead, climate migrants are disproportionately absorbed by agriculture and services. These patterns stand in contrast to frictionless models, which predict that manufacturing should expand in regions receiving labor. 

To shed more light on the mechanisms behind these discrepancies, we exploit matched employer-employee data to measure spatial labor market frictions at the firm level. We show that manufacturing firms are significantly less connected to migrant networks than firms in agriculture or services, largely because manufacturing is geographically concentrated and relies more heavily on local labor markets. As a result, displaced agricultural workers face higher matching frictions when attempting to enter manufacturing jobs outside their origin regions and are more likely to find jobs in smaller firms outside manufacturing. Once we account for these asymmetric spatial frictions, the observed sectoral allocation of migrants aligns closely with the model's predictions.

An alternative explanation for this lack of spatial labor reallocation into manufacturing is that workers displaced by drier climatic conditions -- and especially former agricultural workers -- might not have the skills required to work in manufacturing in destination regions. In this case, the absence of migrant reallocation into manufacturing would not reflect spatial frictions but an optimal allocation of labor at destination. To investigate this mechanism, we split workers by their level of education. We find that low-skill workers are more likely to relocate into the agricultural sector, while high-skill workers are more likely to relocate into services. However, neither low-skill nor high-skill workers relocate into manufacturing, which suggests that labor market frictions play a role. 

Our findings imply that spatial capital and labor market frictions are a major constraint to factor reallocation in response to climate change. The optimal response to lower agricultural productivity would be a reallocation of both factors towards the other traded sector, manufacturing, which is concentrated in space. As a result, a large part of this reallocation process needs to take place across regions. However, we find that spatial capital and labor market frictions constrain spatial factor reallocation towards manufacturing. This limits the ability of developing economies to use migration and capital flows to facilitate structural transformation in response to climate change. Our results thus underscore the importance of policies that ease inter regional labor matching and improve the resilience of financial intermediation. Without such policies, climate change may not only reduce agricultural productivity but also slow structural transformation by misallocating labor and capital across space and sectors.   
