Income inequality and entrepreneurial dynamics are strongly
related (Halvarsson et al., 2018, Packard and Bylund, 2018, Bru-
ton et al., 2021). On the one hand, the rise of superstar innova-
tive entrepreneurs is an important driving force behind rising in-
equality (Gabaix et al., 2016, Aghion et al., 2019). On the other
hand, Braggion et al. (2021) show that inequality negatively af-
fects the entrepreneurial activities of low-income households.2 Re-
latedly, Lee et al. (2022) show that rising income inequality re-
duces job creation in small firms.
How did the COVID-19 pandemic and the subsequent 2020
recession affect the relationship between inequality and en-
trepreneurship? The COVID-19 crisis caused large income losses
for the affected households, and the high level of uncertainty re-
duced access to bank loans for potential entrepreneurs.3 Despite
the prompt policy measures to support incumbent firms hit by the
shock, such as furlough schemes, guaranteed loans or moratoriums,
there has been more limited and less timely government support 
for the creation of new business.4 Since startups play a key role in
terms of job creation, innovation and long-run growth, the lack of
firm creation can hinder the recovery and future growth, generat-
ing a missing generation of firms (Sedlᡠcek, 2020).
The COVID-19 shock nonetheless also presented an opportunity
to open new types of digitally oriented businesses, and the avail-
ability of public subsidies and a large pool of unemployed work-
ers were also factors that should have promoted firm creation (Li-
Ying and Nell, 2020). How did these positive and negative shocks
affect the formation of new businesses? And were these effects
heterogeneous along the income distribution?
Despite their importance, little is known about these issues,
since most of the recent research focuses on the effect of the
COVID-19 shock on incumbent businesses, and relatively few stud-
ies analyze how it has affected new business dynamics. Some de-
scriptive early studies analyzed the overall dynamics of new busi-
ness applications (see, e.g., Dinlersoz et al., 2021 for the US and
Fritsch et al., 2021 for Germany) and net entrepreneurial flows
(e.g., Fairlie, 2020). However, to the best of our knowledge, none
of the existing studies analyzes how the COVID-19 shock heteroge-
neously affected new entrepreneurial households or what the im-
plications were for the types of new startups created.
In this paper, we provide an answer to these questions by an-
alyzing real-time data from the COVID-19 recession. Specifically,
we provide in-depth analysis of a new extensive survey of more
than 24,000 households on their entrepreneurial attitudes and de-
cisions (the 2020 wave of the Global Entrepreneurship Monitor –
GEM – survey for Spain). The data are representative of the whole
adult population of Spain, and rich and detailed enough to al-
low us to disentangle the main drivers of firm creation during
COVID-19 while controlling for individual characteristics. Further-
more, since this survey has been conducted (as a repeated cross
section) since 1999, we can compare our findings to the character-
istics of firm entry during the Great Recession of 2008–2010. Im-
portantly, the 2020 survey, while including all questions consistent
with the surveys in previous years, also includes a set of additional
questions on the COVID-19 recession (for example, asking whether
the households directly suffered a loss of income because of the
pandemic or whether they had new business opportunities), which
allow us to disentangle the different factors driving our results.
Our main findings are as follows. Controlling for population
characteristics (age, gender, income, and education), we find that
the overall decline in the startup rate during the 2020 COVID-19
recession was large and of a similar magnitude to the decline dur-
ing the Great Recession years. Entry declined by approximately 40%
with respect to the long-run average entry rate (1.7 pp). More-
over, the decline in firm entry has been more concentrated among
startups with high growth potential, as also happened during the
Great Recession.5
More importantly, we find that the COVID-19 recession and
the Great Recession present striking differences regarding the im-
pact on households with different income levels. During 2020, the
drop in firm entry was entirely concentrated among low- and
medium-income households. In fact, we find no reduction in en-
try among high-income households (defined as the top tercile in-
come group). Furthermore, the changes in entry composition to-
wards low-growth firms only occur among low- and medium-
income households, while it increases among high-income house-
holds. These results, which hold when we control for education
and age interacted with the recession dummies, are surprising and
at odds with the patterns during the Great Recession, during which
high-income households suffered a stronger decline in entry than
those with medium or low income. The reason for this divergence
is likely to be related to the different nature of the shock: during
the Great Recession, Spain suffered a huge burst in the housing
bubble. Households in the upper tercile of income are more likely
to own real estate, which is often used as collateral for new credit,
and hence they were more affected by this shock. In addition, we
show that these differential results are not driven by an uneven
surge in necessity entrepreneurs among income groups.
Next, we explore whether conditional on starting a firm, en-
trepreneurs with different income levels select different financing
sources, and how such selection changed during COVID-19. Regard-
ing the sources of funds, the last 4 waves of the survey (2017–
2020) include detailed information on the amount of initial financ-
ing of startups and the main sources used to finance them. Con-
sistent with other studies, the survey shows that smaller startups
are mostly internally financed, while larger startups are more in-
tensely bank-financed, a finding consistent with the pecking or-
der of finance. During COVID-19, in addition to observing a re-
duction in startups from low/middle-income households, we also
see that they greatly increased the use of their own savings (rela-
tive to the other sources), while we do not observe the same in-
crease for high-income households. This finding is consistent with
a tightening of financial conditions (in line with the findings of
Ferrando and Ganoulis, 2020), pointing at financial frictions being
important in explaining the different performance of entrepreneurs
along the income distribution. One possible alternative explana-
tion is that instead there were fewer opportunities of smaller size
for low-income entrepreneurs than for high-income entrepreneurs.
Because opportunities were smaller, internal finance was used rela-
tively more than in the pre-COVID years, since the cheapest source
of funds is used first. This alternative story would imply that in-
ternally financed projects should be on average smaller during
COVID-19 than in the pre-COVID-19 years, because such smaller
size is the necessary condition for observing an increase in the rel-
ative use of internal finance. However, our results do not find any
evidence of this. We are also able to rule out that a possible lack
of entrepreneurial skills of low-income households impeded them
from exploiting new business opportunities during the pandemic.
For this, we show that our main findings are robust to only includ-
ing the subset of entrepreneurs that reports having the relevant
skills for opening a business.
We then explore whether the COVID-19 recession, in addi-
tion to negatively affecting the income and wealth of many en-
trepreneurial households, also presented new opportunities, which
were disproportionately taken by high-income households because
of their larger wealth and better access to external finance. We
find empirical evidence consistent with this explanation using de-
tailed information on the type of business created and on the
sources of funds used to create them. We use the unique infor-
mation provided in the GEM surveys, where entrepreneurs de-
scribe in their own words the kind of business they are intend-
ing to create, to precisely identify new business that are digital
and internet-oriented. Our results show that the fraction of digital
businesses increased during COVID-19, and that this increase was
entirely driven by high-income households, for which we observe
a 70 percent increase in digital startups relative to the 2011–2019 period.
As robustness, we check whether these results are driven by
possible confounding factors. One potential explanation for the
findings above is that low- and middle-income households are
more likely to be engaged in sectors directly affected by the pan-
demic (e.g. leisure and hospitality, and transport), or in businesses
that require more face-to-face interaction than the businesses of
high-income households, and therefore are more directly exposed
to the COVID-19 shock. However, we find no evidence supporting
this. Our results show that high-income households did much bet-
ter during COVID-19 than the other households in terms of entry
into entrepreneurship in the non-affected sectors, while we find
a smaller difference in the affected sectors. Furthermore, it seems
that results are not driven by the ‘size’ of the negative income
shock received by the household: the difference between high-
and medium/low-income households was particularly large among
those that did not suffer a negative income shock during the pan-
demic.
Therefore, although we cannot completely rule out possible al-
ternative explanations, a plausible interpretation of our findings is
that low- and medium-income households had similar opportuni-
ties as high-income households, but because of more difficult ac-
cess to external financial resources, many could not exploit them
(more difficulty accessing banks and perhaps also more difficulty
accessing family and friends because these sources increased their
precautionary savings). Thus, while the funding channel is un-
likely to be the sole driver of the large drop in the startup rate
in 2020, these findings highlight that the constrained access to
finance for low- and middle-income entrepreneurial households
was an important reason for why their entrepreneurial activity
was much more negatively affected than that high-income house-
holds.6 Their implication is that, while policies directed at sup-
porting current jobs during a pandemic are important, to ensure
a durable recovery in the future, they should be accompanied by
measures directed at reducing the cost of credit for new potential
entrepreneurs.
Finally, in order to gauge the long-term consequences of these
results on aggregate employment, the last section of the paper
summarizes the findings of Appendix B, where we match GEM sur-
vey data with firm-level panel data from Central de Balances In-
tegrada. We estimate that the employment of the 2020 cohort of
firms is expected to be 2.4 percent smaller after 10 years because of the
“missing generation” of low- and medium-income entrepreneurs.
