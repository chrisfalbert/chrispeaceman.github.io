Is immigration beneficial for native workers because it leads to the creation of
additional jobs, or does it harm their labor market prospects through higher job
competition? This question has been the subject of much debate, as many developed
countries saw rising immigrant inflows over the last few decades. In the United
States, the share of ­ foreign-born residents among the population has increased from
around 5 percent in the 1970s to over 13 percent today, triggered by a change in
immigration policy that facilitated entry from Latin America and Asia. Another
major change in the nature of US immigration since the beginning of the 1990s
is a surge of illegal entries, especially by ­ low-skilled workers. While the number
of all immigrants residing in the United States doubled from around 20 million to
40 million between 1990 and 2013, the number of individuals without legal status
increased almost fourfold from 3 million to over 11 million, of which more than
one-third do not have a high school degree.
The aim of this paper is to shed new light on the distinct effects of documented
and undocumented immigration in the ­ low-skilled labor market. Using US survey
data with legal status imputed following Borjas (2017b), I first document significant
differences in labor market outcomes between natives and immigrants that vary with
legal status. In particular, immigrants have lower wages and higher job finding rates
than natives do, and both these gaps are much larger for undocumented immigrants.
While wage differences that depend on legal status have already been documented
in previous studies (­ Rivera-Batiz 1999, Kossoudji and ­ Cobb-Clark 2002, Pan 2012),
the finding of differences in job finding rates is novel in the literature.
To account for these empirical facts, I extend a standard job search model by a
­ nonrandom hiring mechanism following Barnichon and Zylberberg (2019). Natives
and immigrants are perfect substitutes in production but have different reservation
wages due to heterogeneity in terms of unemployment benefits, bargaining power,
and risk of deportation. Firms can receive multiple applications and choose the can-
didate that they can extract the highest surplus from. This implies higher job finding
rates for workers that accept lower wages and therefore leads to gaps in job finding
rates as in the data. In this model, a rise in the share of immigrant workers leads to
more job creation because it decreases average wage costs. However, it also leads to
higher job competition because firms prefer immigrants over natives. Job creation
and job competition affect the employment rate of natives in opposite ways. Which
of the two effects dominates depends on how large the difference in the wage rates
of natives and immigrants is. The higher the wage costs that firms save by hiring
an immigrant worker, the stronger the job creation effect and the more beneficial
immigration is.
In my empirical analysis, I find that, conditional on observable characteristics,
undocumented immigrants earn 8 percent less and have a 7 percentage points higher
job finding rate than documented immigrants, who in turn earn 4 percent less and
have a 7 percentage points higher job finding rate than natives. I estimate the model
to match these estimates and use it to simulate the labor market effects of both types
of immigration. The simulations show that the job creation effect is large enough to
dominate the job competition effect in the case of undocumented immigration but
not documented immigration. Therefore, only undocumented immigration is pre-
dicted to be unambiguously beneficial for natives as both their employment rate and
wages increase, whereas documented immigration decreases natives’ employment
rate and has an ambiguous effect on wages depending on the assumed wage-bar-
gaining mechanism. I test these predictions from the model empirically by estimat-
ing the effects of immigrant shares in the ­ low-skilled labor force on vacancies and
wages at the metropolitan statistical area (MSA) level. I find positive effects for the
undocumented immigrant share on both vacancies and wages, but I do not find pos-
itive effects for the documented immigrant share. This supports the prediction that
undocumented immigration increases employment opportunities and wages more
than documented immigration does.
Finally, I use the framework to study the impact of a counterfactual policy of
stricter immigration enforcement, which I simulate through an increase in the exog-
enous deportation rate of undocumented immigrants. I distinguish two cases: (i) a
rise in the deportation rate that is the same independent of employment status, and
(ii) a rise in the deportation rate for only employed workers, for example, because
of an intensified use of worksite raids by authorities. In the first case, the policy
leads to a marginal increase in legal workers’ unemployment rates, as expected
firm surplus and thus job creation are dampened weakly. In the second case, firms
­ additionally have to pay a compensating differential to induce an undocumented
worker to accept a job. As a result, wage costs increase and job creation is dampened
more strongly, which implies a much stronger rise in unemployment and a fall in
wages for legal workers. I test these predictions using the ­ statewide implementation
of omnibus immigration laws as a measure of stricter immigration enforcement and
find that introducing these laws is associated with a lower job finding rate for all
workers, which is evidence for muted vacancy creation. Moreover, I find that wages
fall for natives and rise for undocumented immigrants, which is consistent with a
compensating differential in their wages.
This paper contributes to the literature by documenting large job finding rate
­ differences between natives and immigrants, which are inversely related to their
gaps in wages, and by analyzing both documented and undocumented immigration
in a search model that allows firms to choose their preferred applicant among sev-
eral. This simple and intuitive extension enables the model to match the empirical
fact of a variation in job finding rates across heterogeneous workers, which is a
puzzle for the standard model with strictly random hiring.
There exists a large body of related literature investigating the labor market impact
of immigration. Previous studies employed spatial correlations (Card 2001; Glitz
2012; Dustmann, Fasani, and Speciale 2017), skill cell correlations (Borjas 2003,
Mishra 2007), or structural production function approaches (Ottaviano and Peri
2012; Manacorda, Manning, and Wadsworth 2012) to identify wage effects. More
recently, some authors also employ search frameworks (Chassamboulli and Palivos
2014, Battisti et al. 2018), which allows them to study the effects of immigration on
both wages and employment.
In a study that is closely related to this paper, Chassamboulli and Peri (2015)
distinguish between documented and undocumented immigration in a search frame-
work also featuring a job creation effect resulting from the arrival of workers that
accept lower wages. Hiring is random in this model, implying that firms cannot
discriminate between natives and immigrants in their hiring decisions. Therefore, all
workers have the same job finding rates, and immigration, whether documented or
undocumented, unambiguously drives up the wages and employment of natives. The
extension with a ­ nonrandom hiring mechanism that I propose in this paper generates
job finding rate differences consistent with the data and gives rise to the competition
effect, which implies that the beneficiality of immigration depends on the difference
between the wages of natives and the entering immigrant type.
While there are several other studies that distinguish immigrants by legal status
in theoretical models (Liu 2010, Edwards and Ortega 2017, Machado 2017), there
exists little empirical work on the labor market impact of undocumented immi-
grants. Using administrative data from the US state of Georgia, in which undoc-
umented immigrants can be identified through invalid social security numbers,
Brown, Hotchkiss, and ­ Quispe-Agnoli (2013) provide evidence that the employ-
ment of undocumented workers leads to lower exit rates of firms through a compet-
itive advantage. Using the same dataset, Hotchkiss, ­ Quispe-Agnoli, and ­ Rios-Avila
(2015) show that wages of documented workers increase with a higher share of
undocumented workers at both the ­ county-industry and the firm level. These find-
ings of higher firm profits and improved labor market outcomes of natives due to
a higher fraction of undocumented workers are in line with the predictions of the
model proposed in this paper.